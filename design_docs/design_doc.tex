\documentclass{article}
\usepackage{hyperref}
\usepackage{changepage}
\title{acm.mst.edu Design Documentation}
\author{Kevin Schoonover}
\date{\today}

\begin{document}

\maketitle

\tableofcontents

\section{Introduction}
\subsection{Purpose}
In order to start on a large-scale project like the ACM-General website, I
believe that it is important to first talk about why we are doing the project.

The main reason that that I started this process originally was threefold.
Firstly, the majority of the System Administrator job was updating
the old ACM-General website by manually uploading photos and inputting them into
a WordPress site and then creating WordPress events; I personally believed that we
could create a more personalized and intelligent approach which would allow these
repetitive tasks to be more easily automated. Secondly, I wanted to create a 
more succinct and refined user experience for everyone who interacted with the 
website thereby increasing user productivity and activity within ACM. 
Lastly, I believed that the website was capable of so much more than a 
'attractive' flier and an event share. Features like automatically managing SIGs, 
allowing for developer interaction through an API, and much more would allow
for ACM to grow in a more extensive and intelligent way. All of this and 
more will be discussed throughout this Design documentation.

\subsection{Tools / Prerequisite Knowledge}
Some tools, frameworks, and languages one should attempt to familiarize 
themselves with to jump head-first into the project:
\begin{itemize}
    \item Django (\url{https://docs.djangoproject.com/en/1.11/}) - Web 
          Framework and back-end which runs all of the webpage interaction.
    \item Django REST Framework (\url{http://www.django-rest-framework.org/}) -
          A nice library which helps to automatically configure and manage the
          API.
    \item Git (\url{http://www.sphinx-doc.org/en/stable/}) - Source control
          management software which allows for intelligent sharing and
          management of code.
    \item Jenkins (\url{https://jenkins.io/}) - Another continuous
          integration platform which automatically deploys the website.
    \item Python3 (\url{https://docs.python.org/3/}) - The main programing
          language used in essentially everything we write.
    \item Sphinx (\url{http://www.sphinx-doc.org/en/stable/}) - Automatic
          documentation suite which is a python standard.
    \item Stripe (\url{https://stripe.com/}) - Payment gateway in which all of
          the payments will run through.
    \item Travis CI (\url{https://travis-ci.org/}) - Continuous integration
          platform which runs the automated testing of the code base.
\end{itemize}

\section{Design}
\subsection{Web API}
In order to improve overall modularity and extensibility, a RESTful Web API
exposing the majority, if not all, of the Django models must be implemented. 

\subsubsection{Schema}
The URL scheme for the web-api will be as follows: 
\\ \texttt{http://www.foobar.com/web-api/<api\_version>/<model\_name>/(model\_pk)}
\\ 
\\
\textbf{api\_version} (Required): Whatever version of the API that is used. All
previous versions of the API will be reserved (unless expressly deprecated to
all users of the API) in order to preserve backwards compatibility of software
which may be legacy. The most current version will be named \texttt{latest}
% Full Indent
\\
\textbf{model\_name} (Required): The name of the model which the user is
attempting to access with the API call with every character lowercase. 
% Full Indent
\\
\textbf{model\_pk} (Optional): The specific primary key of the model instance
the user is attempting to access. % Full Indent

\subsubsection{Permissions}
Any developer wishing to gain access to the API will need to be registered on 
the site. After the user log-in, they can visit the route 
\textbf{/developer/api\_keys} which displays a hierarchy of all models and
permissions (GET (all), GET (specific primary key), POST, PUT, DELETE) for each
model. The user can then request any level of access they require (including
all API permissions) by selecting which permissions they want on a per-model
basis. After the user submits their request for permissions, the system
administrator or any person responsible for managing API keys can confirm or
deny the key request for that person. \textbf{TBD: Internal messaging system or
email}. %TODO

\subsubsection{Transactions}
\texttt{/web-api/<api\_version>/<model\_name>/}:
\\
\textbf{GET}: Returns a JSON of all existing \texttt{<model\_name>} references in the
database.
\\
\textbf{POST}: Creates a new \texttt{<model\_name>} with the properties specified in the
POST request.
\\
\textbf{OPTIONS}: Returns all the possible options of the endpoint.

\subsection{Events}
\subsection{Payments}
\subsection{SIGs}
\subsection{Users}
\subsection{Sponsors}
\subsection{External Developers}
\subsection{Permissions}
\subsubsection{Permissions ideas}
Brainstorming of all permission ideas necessary for the website. Section will be 
refined/removed later.
\begin{itemize}
    \item Manage API Keys
    \item Per-model
    \begin{itemize}
        \item GET specific instance
        \item GET all models
        \item DELETE specific instance
        \item Create new instance
        \item Modify existing instance 
    \end{itemize}

    \item Per-SIG
    \begin{itemize}
        \item Manage SIG roles / permissions
        \item Create SIG events
        \item Manage SIG members
        \item Manage SIG budget?
    \end{itemize}

    \item Per-view (need to have a better way of modulating this)
    \begin{itemize}
        \item Can GET
        \item Can POST 
    \end{itemize}


\end{itemize}
\subsection{Testing}
\subsection{Permissions}

\section{Workflow}

\section{Milestones / Deadlines}

\end{document}
